%% LyX 2.1.4 created this file.  For more info, see http://www.lyx.org/.
%% Do not edit unless you really know what you are doing.
\documentclass[english]{article}
\usepackage[T1]{fontenc}
\usepackage[latin9]{inputenc}
\usepackage{array}
\usepackage{float}
\usepackage{multirow}
\usepackage{amstext}

\makeatletter

%%%%%%%%%%%%%%%%%%%%%%%%%%%%%% LyX specific LaTeX commands.
%% Because html converters don't know tabularnewline
\providecommand{\tabularnewline}{\\}

\makeatother

\usepackage{babel}
\begin{document}

\title{Effective trilinear channels For Git Test }


\author{Boris Panes}

\maketitle

\section{Gravitino decay channels in trilinear RpV}

Formally, the model that we are considering is supersymmetry with
trilinear R-parity violation in the leptonic sector and also we consider
the gravitino as the lightest susy particle and dark matter candidate.
Since R-parity is broken our DM candidate is metastable and could
be detected indirectly. 

The lagrangian of this model must be investigated carefully in order
to understand the relationship between the free parameters of the
model and the strengths of gravitino decays, which is something that
we are currently working on and should be added to our notes for the
paper. However, for the current discussion we just need to know that
in principle we can have the following decay channels $G\rightarrow l_{i}^{+}l_{j}^{-}\nu_{k}$
with $i,j,k$ running over the three SM families of leptons, thus
in principle we have 27 different final state channels to consider.

Fortunately, for the fit of measured charged leptons from AMS-02,
CALET or DAMPE, we can consider only the gravitino final state channels
that in principle could produce different spectra of electrons and
positrons. Therefore, until now, we have simulated with Pythia and
used for the fit the list of gravitino decay channels shown in table
\ref{tab:Independent-channels-prompt-final-states}, which seem to
be independent when we consider the nature of the produced prompt
leptons after decay.

\begin{table}
\centering{}%
\begin{tabular}{|c|c|c|c|}
\hline 
Channel & Final State & Details & Acronym\tabularnewline
\hline 
\hline 
1 & $e\text{\textsuperscript{+}}\mu\text{\textsuperscript{-}}\nu$ & antielectron-muon-neutrino & AEMuNue\tabularnewline
\hline 
2 & $e\text{\textsuperscript{+}}\tau\text{\textsuperscript{-}}\nu$ & antielectron-tau-neutrino & AETauNue\tabularnewline
\hline 
3 & $e\text{\textsuperscript{-}}\mu\text{\textsuperscript{+}}\nu$ & electron-antimuon-neutrino & EAMuNue\tabularnewline
\hline 
4 & $\mu\text{\textsuperscript{-}}\mu\text{\textsuperscript{+}}\nu$ & muon-antimuon-neutrino & MuAMuNue\tabularnewline
\hline 
5 & $\tau\text{\textsuperscript{-}}\tau\text{\textsuperscript{+}}\nu$ & tau-antitau-neutrino & TauATauNue\tabularnewline
\hline 
6 & $\tau\text{\textsuperscript{-}}\mu\text{\textsuperscript{+}}\nu$ & tau-antimuon-neutrino & TauAMuNue\tabularnewline
\hline 
7 & $\tau\text{\textsuperscript{+}}\mu\text{\textsuperscript{-}}\nu$ & antitau-muon-neutrino & ATauMuNue\tabularnewline
\hline 
8 & $e\text{\textsuperscript{-}}e\text{\textsuperscript{+}}\nu$ & electron-antielectron-neutrino & EAENue\tabularnewline
\hline 
9 & $e\text{\textsuperscript{-}}\tau\text{\textsuperscript{+}}\nu$ & electron-antitau-neutrino & EATauNue\tabularnewline
\hline 
\end{tabular}\caption{\label{tab:Independent-channels-prompt-final-states}Independent channels
considering prompt final states. Notice that we use $\nu$ to indicate
any flavor of neutrinos. }
\end{table}


Thus, in principle we have 8 independent branching ratios, which is
a bunch of free parameters to add to the fit of charged leptons. Recall
that we also have 4 parameters from the modeling of the source, 2
to 3 from the modeling of systematic uncertainties and 2 more associated
to the mass and lifetime of the gravitinos. However, by studying the
propagated spectra of electrons and positrons, we have noticed that
there are channels with different final state particles that produce
basically the same spectrum of charged leptons, which we consider
as coincident spectra (this is not necessarily related to charge conjugation).
For instance we obtain the same spectrum of positrons from the channels
$e\text{\textsuperscript{+}}\mu\text{\textsuperscript{-}}\nu$ and
$e\text{\textsuperscript{+}}\tau\text{\textsuperscript{-}}\nu$, which
can be understood because $\mu\text{\textsuperscript{-}}$ and $\tau\text{\textsuperscript{-}}$
do not produce extra positrons. The independent spectrum groups are
shown in table \ref{tab:Independent-channels-spectrum}.

\begin{table}
\begin{centering}
\begin{tabular}{|c|c|c|}
\hline 
Charged lepton spectrum & Group ID & Channels\tabularnewline
\hline 
\hline 
\multirow{3}{*}{Electron} & g1-ele & 1-4-7\tabularnewline
\cline{2-3} 
 & g2-ele & 2-5-6\tabularnewline
\cline{2-3} 
 & g3-ele & 3-8-9\tabularnewline
\hline 
\multirow{3}{*}{Positron} & g1-pos & 1-2-8\tabularnewline
\cline{2-3} 
 & g2-pos & 3-4-6\tabularnewline
\cline{2-3} 
 & g3-pos & 5-7-9\tabularnewline
\hline 
\end{tabular}
\par\end{centering}

\caption{\label{tab:Independent-channels-spectrum}Independent spectrum groups
considering the coincidence between the propagated spectrum of charged
leptons.}
\end{table}


Besides, from consistency, the spectrum of electrons from one channel
$A$ must be equal to the spectrum of positrons from the charge conjugated
one $\bar{A}$. This is not necessarily an assumption about the coupling
strengths of the lagrangian, since this only relies in the equivalence
between charge conjugated channels and their corresponding final states.
For instance, we should have that the electron spectrum from $e\text{\textsuperscript{+}}\mu\text{\textsuperscript{-}}\nu$
coincides with positron one from $e\text{\textsuperscript{-}}\mu\text{\textsuperscript{+}}\nu$,
etc. The list of equivalent spectra are shown in table \ref{tab:Equivalence-between-AAbar},
where we have used the groups defined in table \ref{tab:Independent-channels-spectrum}.

\begin{table}
\begin{centering}
\begin{tabular}{|c|c|}
\hline 
Electron Spectrum & Positron Spectrum\tabularnewline
\hline 
\hline 
g1-ele & g2-pos\tabularnewline
\hline 
g2-ele & g3-pos\tabularnewline
\hline 
g3-ele & g1-pos\tabularnewline
\hline 
\end{tabular}
\par\end{centering}

\caption{\label{tab:Equivalence-between-AAbar}Equivalence between the electron
spectrum of channel $A$ and the positron spectrum of channel $\bar{A}$.}
\end{table}


Finally, we also have some arguments (to be properly developed starting
from the lagrangian of the model and using c.c. invariance) to assume
that there is an equivalence between the branching fractions associated
to \textit{charge conjugated} channels. So, in this case we are basically
assuming that the branching fraction of the $e\text{\textsuperscript{+}}\mu\text{\textsuperscript{-}}\nu$
channel is equal to the branching fraction of the $e\text{\textsuperscript{-}}\mu\text{\textsuperscript{+}}\nu$
one. We summarize these conditions in table \ref{tab:Equivalence-between-branching-fractions}.
The details about this assumption are going to be released as soon
as possible

\begin{table}[H]
\begin{centering}
\begin{tabular}{|c|c|}
\hline 
BR of Channel $A$ & BR of Channel $B$\tabularnewline
\hline 
\hline 
1 ($e\text{\textsuperscript{+}}\mu\text{\textsuperscript{-}}\nu$) & 3 ($e\text{\textsuperscript{-}}\mu\text{\textsuperscript{+}}\nu$)\tabularnewline
\hline 
2 ($e\text{\textsuperscript{+}}\tau\text{\textsuperscript{-}}\nu$) & 9 ($e\text{\textsuperscript{-}}\tau\text{\textsuperscript{+}}\nu$)\tabularnewline
\hline 
6 ($\tau\text{\textsuperscript{-}}\mu\text{\textsuperscript{+}}\nu$) & 7 ($\tau\text{\textsuperscript{+}}\mu\text{\textsuperscript{-}}\nu$)\tabularnewline
\hline 
\end{tabular}
\par\end{centering}

\caption{\label{tab:Equivalence-between-branching-fractions}Equivalence between
branching fractions motivated from c.c. invariance (we have to check
this considering the real gravitino-trilinear lagrangian).}
\end{table}



\section{Effective flux of charged leptons}

Here, we are going to use the results of the previous section to rewrite
the flux of charged leptons. First, let us recall that the flux of
electrons from gravitino decays can be written as

\begin{equation}
\frac{dN(e\text{\textsuperscript{-}})}{dE}\propto\frac{1}{m_{G}\tau_{G}}\sum_{i=1}^{9}BR_{i}\frac{dN_{i}(e\text{\textsuperscript{-}})}{dE}\label{eq:ch-lepton-spec}
\end{equation}


\noindent where $m_{G}$ is the gravitino mass, $\tau_{G}$ is the
gravitino lifetime, the index $i$ cover the 9 channels of table \ref{tab:Independent-channels-prompt-final-states},
$BR_{i}$ are the corresponding branching ratios of each channel and
$dN_{i}(e\text{\textsuperscript{-}})/dE$ are the propagated gravitino
decay spectra (normalized to one decay), which are obtained from the
propagation of charged leptons through the galactic environment. Here
we need to unify conventions with German's notation but in principle
is just a matter of names, nothing deep.

In principle, we can get each $BR{}_{i}$ as a function of the free
parameters of our model, such as the trilinear couplings $\lambda_{ijk}$
or the mass of scalars (at some point we have to do this computation,
for which we suggest to follow hep-ph/0107286), but in general we
can consider the branching fractions as the effective free parameters
for the fit of charged lepton measurements with the condition that
$\sum_{i}BR_{i}=1$. Furthermore, considering the results of table
\ref{tab:Independent-channels-spectrum} we can group-factorize some
spectra and write that

\begin{eqnarray*}
\frac{dN(e\text{\textsuperscript{-}})}{dE} & \propto & \frac{1}{m_{G}\tau_{G}}\biggl[(BR_{1}+BR_{4}+BR_{7})\frac{dN_{1}(e\text{\textsuperscript{-}})}{dE}+\\
 &  & \,\,\,\,\,\,\,\,\,\,\,\,\,\,\,\,\,\,\,(BR_{2}+BR_{5}+BR_{6})\frac{dN_{2}(e\text{\textsuperscript{-}})}{dE}+\\
 &  & \,\,\,\,\,\,\,\,\,\,\,\,\,\,\,\,\,\,\,(BR_{3}+BR_{8}+BR_{9})\frac{dN_{3}(e\text{\textsuperscript{-}})}{dE}\biggr]\\
\frac{dN(e\text{\textsuperscript{-}})}{dE} & \propto & \frac{1}{m_{G}\tau_{G}}\biggl[\alpha_{1}\frac{dN_{1}(e\text{\textsuperscript{-}})}{dE}+\alpha_{2}\frac{dN_{2}(e\text{\textsuperscript{-}})}{dE}+\alpha_{3}\frac{dN_{3}(e\text{\textsuperscript{-}})}{dE}\biggr]
\end{eqnarray*}


\noindent where $\alpha_{1}=BR{}_{1}+BR_{4}+BR_{7}$, $\alpha_{2}=BR{}_{2}+BR_{5}+BR_{6}$
and $\alpha_{3}=BR{}_{3}+BR_{8}+BR_{9}$ with $\alpha_{1}+\alpha_{2}+\alpha_{3}=1$.
Thus, we just need to define two independent effective branching fractions
for the fit of electrons. Similarly, for the positron spectrum we
have that

\begin{eqnarray*}
\frac{dN(e\text{\textsuperscript{+}})}{dE} & \propto & \frac{1}{m_{G}\tau_{G}}\biggl[(BR_{1}+BR_{2}+BR_{8})\frac{dN_{1}(e\text{\textsuperscript{+}})}{dE}+\\
 &  & \,\,\,\,\,\,\,\,\,\,\,\,\,\,\,\,\,\,\,(BR_{3}+BR_{4}+BR_{6})\frac{dN_{3}(e\text{\textsuperscript{+}})}{dE}+\\
 &  & \,\,\,\,\,\,\,\,\,\,\,\,\,\,\,\,\,\,\,(BR_{5}+BR_{7}+BR_{9})\frac{dN_{5}(e\text{\textsuperscript{+}})}{dE}\biggr]
\end{eqnarray*}


Besides, we can use the branching ratio equivalences shown in table
\ref{tab:Equivalence-between-branching-fractions} to rewrite the
positron spectrum as

\begin{eqnarray*}
\frac{dN(e\text{\textsuperscript{+}})}{dE} & \propto & \frac{1}{m_{G}\tau_{G}}\biggl[(BR_{3}+BR_{2}+BR_{8})\frac{dN_{1}(e\text{\textsuperscript{+}})}{dE}+\\
 &  & \,\,\,\,\,\,\,\,\,\,\,\,\,\,\,\,\,\,\,(BR_{1}+BR_{4}+BR_{7})\frac{dN_{3}(e\text{\textsuperscript{+}})}{dE}\\
 &  & \,\,\,\,\,\,\,\,\,\,\,\,\,\,\,\,\,\,\,(BR_{2}+BR_{5}+BR_{6})\frac{dN_{5}(e\text{\textsuperscript{+}})}{dE}\biggr]\\
\frac{dN(e\text{\textsuperscript{+}})}{dE} & \propto & \frac{1}{m_{G}\tau_{G}}\biggl[\alpha_{1}\frac{dN_{3}(e\text{\textsuperscript{+}})}{dE}+\alpha_{2}\frac{dN_{5}(e\text{\textsuperscript{+}})}{dE}+\alpha_{3}\frac{dN_{1}(e\text{\textsuperscript{+}})}{dE}\biggr]
\end{eqnarray*}


Finally, we use table \ref{tab:Equivalence-between-AAbar} to relate
the positron spectrum to the electron one

\begin{eqnarray*}
\frac{dN(e\text{\textsuperscript{+}})}{dE} & \propto & \frac{1}{m_{G}\tau_{G}}\biggl[\alpha_{1}\frac{dN_{1}(e\text{\textsuperscript{-}})}{dE}+\alpha_{2}\frac{dN_{2}(e\text{\textsuperscript{-}})}{dE}+\alpha_{3}\frac{dN_{3}(e\text{\textsuperscript{-}})}{dE}\biggr]\\
\frac{dN(e\text{\textsuperscript{+}})}{dE} & \equiv & \frac{dN(e\text{\textsuperscript{-}})}{dE}
\end{eqnarray*}


Therefore, for the fit of AMS-02, CALET or DAMPE we just need two
$\alpha's$ and three independent spectra. Furthermore, we get automatically
the electron-positron symmetry for (gravitino) dark matter decays
which is expected from general arguments considering charge conjugation
symmetry.


\section{Gamma rays at Fermi-LAT}

As we only can fix $\alpha_{1}$, $\alpha_{2}$ and $\alpha_{3}$
from the fit of charged leptons we have some freedom to choose the
individual branching fractions to generate the photon flux. Also we
must notice that for the photon spectrum we do not have coincidences
between the spectrum of different channels such as $e\text{\textsuperscript{+}}\mu\text{\textsuperscript{-}}\nu$
and $e\text{\textsuperscript{+}}\tau\text{\textsuperscript{-}}\nu$,
as we had for positrons. Therefore we are free to choose $BR{}_{1}$
to $BR_{9}$ with the conditions that
\begin{eqnarray*}
\alpha_{1} & = & BR{}_{1}+BR_{4}+BR_{7}\\
\alpha_{2} & = & BR{}_{2}+BR_{5}+BR_{6}\\
\alpha_{3} & = & BR{}_{3}+BR_{8}+BR_{9}
\end{eqnarray*}


In order to decrease the possible number of photons to be produced
we may choose $BR_{i}$ in the following way,

\begin{eqnarray*}
BR_{1} & = & \alpha_{1},\,\,\,BR_{4}=BR_{7}=0\\
BR_{2} & = & \alpha_{2},\,\,\,BR_{5}=BR_{6}=0\\
BR_{8} & = & \alpha_{3},\,\,\,BR_{3}=BR_{9}=0
\end{eqnarray*}


This choice needs to be justified but in principle it has allowed
us to find compatible points, which can be seen from the plots of
German-Jul17. 
\end{document}
\grid
